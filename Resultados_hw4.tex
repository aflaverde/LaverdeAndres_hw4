\documentclass{article}

%paquetes
\usepackage[utf8]{inputenc}
\usepackage[spanish]{babel}
\usepackage{graphicx}
\usepackage{multicol}
\usepackage{float}

\begin{document}
\title{Resultados Tarea 4}
\author{Andrés Felipe Laverde Martínez}
\maketitle
%%%%%%%%%%%% ODES GRAFICAS
\section{ODE's}

\begin{figure}

\end{figure}

%%%%%%%%%%%% PDES GRAFICAS
\section{PDE's}

\begin{figure}[htb]
	\includegraphics[width=\linewidth]{PDE1.png}
		\caption{Grafico 1 de PDE. Condiciones de frontera fijas a 10 grados Celsius}
\end{figure}

\begin{figure}[htb]
	\includegraphics[width=\linewidth]{PDE2.png}
		\caption{Grafico 2 de PDE. Condiciones de frontera abiertas.}
\end{figure}

\begin{figure}[htb]
	\includegraphics[width=\linewidth]{PDE3.png}
		\caption{Grafico 3 de PDE. Condiciones de frontera periodicas.}
\end{figure}
En todas las gráficas se muestra un avance en el tiempo en la dirección en y (hacia abajo) en el centro de la varilla, con condiciones iniciales arriba del plot.

\end{document}
